%% LyX 2.1.3 created this file.  For more info, see http://www.lyx.org/.
%% Do not edit unless you really know what you are doing.
\documentclass[spanish]{article}
\usepackage[T1]{fontenc}
\usepackage{color}

\makeatletter
%%%%%%%%%%%%%%%%%%%%%%%%%%%%%% User specified LaTeX commands.
\usepackage{babel}

\makeatother

\usepackage{babel}
\addto\shorthandsspanish{\spanishdeactivate{~<>}}

\begin{document}

\title{\textbf{Práctica 7}}


\author{Pablo Cabeza García y Diego González Domínguez}

\maketitle

\section*{Ejercicio:}

\textbf{Dar el pseudo código de un algoritmo incremental que calcule
la envoltura convexa de n puntos en el plano con coste O(n·log n),
justificando la respuesta.}

Los pasos a seguir en el algoritmo incremental serían los siguientes:
\begin{enumerate}
\item Ordenar los puntos en función de su primera coordenada (de izquierda
a derecha).
\item Calcular la envoltura superior y la envoltura inferior.
\item Concatenar ambos resultados
\end{enumerate}
El pseudocódigo sería:

\textcolor{white}{a}

Entrada: Un conjunto P de puntos en el plano. 

Salida: Una lista EC que contiene los vértices de la envoltura convexa
en el sentido horario. 

\textcolor{white}{a}

$P=p_{1},...,p_{n}$ \textcolor{green}{//Ordenar los puntos por la
x-coordenada}

\textcolor{green}{//Cálculo de la envoltura superior:}

$ECs=[p_{1},p_{2}]$ \textcolor{green}{//Insertar $p_{1}$ y $p_{2}$en
la envoltura superior}

for $(3\leq i\leq n)$:

\qquad{}$ECs=[p_{1},...,p_{h-1},p_{h},p_{i}]$ /\textcolor{green}{/Añadir
$p_{i}$ a $ECs$ }

\qquad{}$while$ (los 3 últimos puntos de ECs hacen un giro a la
izquierda):

\qquad{}\qquad{}$ECs=[p_{1},...,p_{h-1},p_{i}]$\textcolor{green}{{}
//Eliminar penúltimo punto de ECs}

\textcolor{green}{//Cálculo de la envoltura inferior:}

$ECi=[p_{n},p_{n-1}]$ \textcolor{green}{//Insertar $p_{n-1}$ y $p_{n}$en
la envoltura inferior}

for $(n-2\geq i\geq1)$:

\qquad{}$ECi=[p_{n},...,p_{h-1},p_{h},p_{i}]$ \textcolor{green}{//Añadir
$p_{i}$ a $ECi$ }

\qquad{}$while$ (los 3 últimos puntos de ECi hacen un giro a la
izquierda):

\qquad{}\qquad{}$ECi=[p_{n},...,p_{h-1},p_{i}]$ \textcolor{green}{//Eliminar
penúltimo punto de ECi}

\textcolor{green}{//Cálculo de la envoltura convexa:}

$ECi=[pi_{1},pi_{2},...,pi_{m-1},pi_{m}]\quad\rightarrow\quad ECi=[pi_{2},...,pi_{m-1}]$
\textcolor{green}{//Eliminar primer y último punto de la envoltura
inferior}

$EC=ECs+ECi$ \textcolor{green}{//Concatenar $ECs$ y $ECi$ y devolver
el resultado}
\end{document}
