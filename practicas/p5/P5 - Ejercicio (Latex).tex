%% LyX 2.1.3 created this file.  For more info, see http://www.lyx.org/.
%% Do not edit unless you really know what you are doing.
\documentclass[spanish]{article}
\usepackage[T1]{fontenc}
\usepackage{color}
\usepackage{amstext}
\usepackage{babel}
\addto\shorthandsspanish{\spanishdeactivate{~<>}}

\begin{document}

\title{\textbf{Práctica 5}}


\author{Pablo Cabeza García y Diego González Domínguez}

\maketitle

\section*{Ejercicio:}

\textbf{1) Si $f=gh$, demuéstrese la fórmula de Leibniz para las
diferencias divididas:}

\begin{equation}
\left[\tau_{i},...,\tau_{i+k}\right]f=\sum{}_{r=i}^{i+k}\left(\left[\tau_{i},...,\tau_{r}\right]g\right)\text{·}\left(\left[\tau_{r},...,\tau_{i+k}\right]h\right)
\end{equation}


Para demostrar la fórmula anterior se utiliza inducción sobre \textit{\textcolor{black}{k}},
de manera que para $k=0$ se tiene:
\begin{equation}
\left[\tau_{i}\right]f=\left[\tau_{i}\right]\left(g\text{·}h\right)=\left(\left[\tau_{i}\right]g\right)\text{·}\left(\left[\tau_{i}\right]h\right)
\end{equation}


En el paso inductivo, si se supone que \textbf{(1)} es cierto para
$k$, y sabiendo que en $k+1$ se cumple la siguiente fórmula:

\begin{equation}
\left[\tau_{i},....,\tau_{i+k+1}\right]f=\frac{\left[\tau_{i},...,\tau_{i+k}\right]f-\left[\tau_{i+1},...,\tau_{i+k+1}\right]f}{\tau_{i}-\tau_{i+k+1}}
\end{equation}


se puede escribir de la siguiente forma:

\[
\left(\tau_{i}-\tau_{i+k+1}\right)\text{·}\left[\tau_{i},...,\tau_{i+k+1}\right]f=\left[\tau_{i},...,\tau_{i+k}\right]f-\left[\tau_{i+1},...,\tau_{i+k+1}\right]f=
\]


\[
=\sum{}_{r=i}^{i+k}\left(\left[\tau_{i},...,\tau_{r}\right]g\right)\text{·}\left(\left[\tau_{r},...,\tau_{i+k}\right]h\right)-
\]


\begin{equation}
-\sum{}_{r=i}^{i+k}\left(\left[\tau_{i+1},...,\tau_{r+1}\right]g\right)\text{·}\left(\left[\tau_{r+1},...,\tau_{i+k+1}\right]h\right)
\end{equation}


Por otro lado, se pueden expresar:

\begin{equation}
\left[\tau_{i+1},...,\tau_{r+1}\right]g=\left[\tau_{i},...,\tau_{r}\right]g-\left(\tau_{i}-\tau_{i+1}\right)\text{·}\left[\tau_{i},...,\tau_{r+1}\right]g
\end{equation}


\begin{equation}
\left[\tau_{r+1},...,\tau_{i+k+1}\right]h=\left[\tau_{r},...,\tau_{i+k}\right]h-\left(\tau_{r}-\tau_{i+k+1}\right)\text{·}\left[\tau_{r},...,\tau_{i+k+1}\right]h
\end{equation}


Sustituyendo \textbf{(5)} y \textbf{(6)} en \textbf{(4)}:

\[
\left(\tau_{i}-\tau_{i+k+1}\right)\text{·}\left[\tau_{i},...,\tau_{i+k+1}\right]f=
\]


\[
=\sum{}_{r=i}^{i+k}\left(\left[\tau_{i},...,\tau_{r}\right]g\right)\text{·}\left(\left[\tau_{r},...,\tau_{i+k}\right]h\right)-\sum{}_{r=i}^{i+k}\left(\left[\tau_{i},...,\tau_{r}\right]g\right)\text{·}\left(\left[\tau_{r},...,\tau_{i+k}\right]h\right)+
\]


\[
+\sum{}_{r=i}^{i+k}\left(\tau_{r}-\tau_{i+k+1}\right)\text{·}\left(\left[\tau_{i},...,\tau_{r}\right]g\right)\text{·}\left(\left[\tau_{r},...,\tau_{i+k+1}\right]h\right)+
\]


\[
+\sum{}_{r=i+1}^{i+k+1}\left(\tau_{i}-\tau_{r}\right)\text{·}\left(\left[\tau_{i},...,\tau_{r}\right]g\right)\text{·}\left(\left[\tau_{r},...,\tau_{i+k+1}\right]h\right)=
\]


\begin{equation}
=\left(\tau_{i}-\tau_{i+k+1}\right)\text{·}\sum{}_{r=i}^{i+k+1}\left(\left[\tau_{i},...,\tau_{r}\right]g\right)\text{·}\left(\left[\tau_{r},...,\tau_{i+k+1}\right]h\right)
\end{equation}


Simplificando queda:

\begin{equation}
\left[\tau_{i},...,\tau_{i+k+1}\right]f=\sum{}_{r=i}^{i+k+1}\left(\left[\tau_{i},...,\tau_{r}\right]g\right)\text{·}\left(\left[\tau_{r},...,\tau_{i+k+1}\right]h\right)
\end{equation}


que es igual a \textbf{(1)} sustituyendo $k$ por $k+1$, y de esta
manera queda demostrada la fórmula de Leibniz.


\section*{\textit{\textcolor{white}{\normalsize{}a}}}

\textbf{2) En el caso particular que $g(x)=x$ y $h(x)=1/x$:}

\[
\left[\tau_{1}\right]g=g(\tau_{1})=\tau_{1}\:\rightarrow\:\left[\tau_{1},\tau_{2}\right]g=\frac{g(\tau_{1})-g(\tau_{2})}{\tau_{1}-\tau_{2}}=\frac{\tau_{1}-\tau_{2}}{\tau_{1}-\tau_{2}}=1\:\rightarrow\:
\]


\[
\:\rightarrow\:\left[\tau_{1},\tau_{2},\tau_{3}\right]g=\frac{\left[\tau_{1},\tau_{2}\right]g-\left[\tau_{2},\tau_{3}\right]g}{\tau_{1}-\tau_{3}}=\frac{1-1}{\tau_{1}-\tau_{3}}=0\:\rightarrow\:...
\]


\begin{equation}
...\:\rightarrow\:\left[\tau_{1},...,\tau_{n}\right]g=0
\end{equation}


\[
\left[\tau_{1}\right]h=h(\tau_{1})=\frac{1}{\tau_{1}}\:\rightarrow\:\left[\tau_{1},\tau_{2}\right]h=\frac{h(\tau_{1})-h(\tau_{2})}{\tau_{1}-\tau_{2}}=\frac{\frac{1}{\tau_{1}}-\frac{1}{\tau_{2}}}{\tau_{1}-\tau_{2}}=\frac{\frac{\tau_{2}-\tau_{1}}{\tau_{1}\text{·}\tau_{2}}}{\tau_{1}-\tau_{2}}=\frac{-1}{\tau_{1}\text{·}\tau_{2}}\:\rightarrow\:
\]


\[
\:\rightarrow\:\left[\tau_{1},\tau_{2},\tau_{3}\right]h=\frac{\left[\tau_{1},\tau_{2}\right]h-\left[\tau_{2},\tau_{3}\right]h}{\tau_{1}-\tau_{3}}=\frac{\frac{-1}{\tau_{1}\text{·}\tau_{2}}-\frac{-1}{\tau_{2}\text{·}\tau_{3}}}{\tau_{1}-\tau_{3}}=\frac{\frac{-\tau_{3}+\tau_{1}}{\tau_{1}\text{·}\tau_{2}\text{·}\tau_{3}}}{\tau_{1}-\tau_{3}}=\frac{1}{\tau_{1}\text{·}\tau_{2}\text{·}\tau_{3}}\:\rightarrow\:...
\]


\begin{equation}
...\:\rightarrow\:\left[\tau_{1},...,\tau_{n}\right]h=\frac{\left(-1\right)^{n-1}}{\tau_{1}\text{·}...\text{·}\tau_{n}}
\end{equation}


\[
\left[\tau_{1}\right]f=f(\tau_{1})=1\:\rightarrow\:\left[\tau_{1},\tau_{2}\right]f=\frac{f(\tau_{1})-f(\tau_{2})}{\tau_{1}-\tau_{2}}=\frac{1-1}{\tau_{1}-\tau_{2}}=0\:\rightarrow\:
\]


\[
\:\rightarrow\:\left[\tau_{1},\tau_{2},\tau_{3}\right]f=\frac{\left[\tau_{1},\tau_{2}\right]f-\left[\tau_{2},\tau_{3}\right]f}{\tau_{1}-\tau_{3}}=\frac{0-0}{\tau_{1}-\tau_{3}}=0\:\rightarrow\:...
\]


\begin{equation}
...\:\rightarrow\:\left[\tau_{1},...,\tau_{n}\right]f=0
\end{equation}


Utilizando los resultados de \textbf{(9)}, \textbf{(10)} y \textbf{(11)}
se comprueba que se cumple \textbf{(1)} para $f(x)=1$, $g(x)=x$
y $h(x)=1/x$:

\begin{equation}
\left[\tau_{1}\right]f=\left(\left[\tau_{1}\right]g\right)\text{·}\left(\left[\tau_{1}\right]h\right)=g(\tau_{1})\text{·}h(\tau_{1})=\tau_{1}\text{·}\frac{1}{\tau_{1}}=1=f(\tau_{1})=\left[\tau_{1}\right]f
\end{equation}


\[
\left[\tau_{1},...,\tau_{n}\right]f=\sum{}_{r=1}^{n}\left(\left[\tau_{1},...,\tau_{r}\right]g\right)\text{·}\left(\left[\tau_{r},...,\tau_{n}\right]h\right)=
\]


\[
=\tau_{1}\text{·}\frac{\left(-1\right)^{n-1}}{\tau_{1}\text{·...·}\tau_{n}}+1\text{·}\frac{\left(-1\right)^{n-2}}{\tau_{2}\text{·...·}\tau_{n}}+0+...+0=
\]


\begin{equation}
=\frac{1-1}{\tau_{2}\text{·...·}\tau_{n}}=0=\left[\tau_{1},...,\tau_{n}\right]f\quad\forall n\geq2
\end{equation}

\end{document}
