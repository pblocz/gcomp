%% LyX 2.1.3 created this file.  For more info, see http://www.lyx.org/.
%% Do not edit unless you really know what you are doing.
\documentclass[spanish]{article}
\usepackage[T1]{fontenc}
\usepackage{amstext}
\usepackage{babel}
\addto\shorthandsspanish{\spanishdeactivate{~<>}}

\begin{document}

\title{\textbf{Práctica 3}}


\author{Pablo Cabeza García y Diego González Domínguez}

\maketitle

\section*{Ejercicio:}

\textbf{Demuéstrese que el máximo de los polinomios de Bernstein $B_{i}^{n}(t)$
con $t\epsilon\left[0,1\right]$ se alcanza en $t=\frac{i}{n}$:}

Los polinomios de Bernstein se pueden escribir como:

\begin{center}
$B_{i}^{n}(t)=\left(\begin{array}{c}
n\\
i
\end{array}\right)\text{·}t^{i}\text{·\ensuremath{\left(1-t\right)}}^{n-i}$
\par\end{center}

Derivando en función de $t$:

\begin{center}
$B_{i}^{n}(t)'=\left(\begin{array}{c}
n\\
i
\end{array}\right)\text{·}\left[i\text{·}t^{i-1}\text{·\ensuremath{\left(1-t\right)}}^{n-i}-t^{i}\text{·\ensuremath{\left(n-i\right)\text{·}\left(1-t\right)}}^{n-i-1}\right]=n\text{·}\left(\begin{array}{c}
n-1\\
i-1
\end{array}\right)\text{·}t^{i-1}\text{·\ensuremath{\left(1-t\right)}}^{n-i}-n\text{·}\left(\begin{array}{c}
n-1\\
i
\end{array}\right)\text{·}t^{i}\text{·\ensuremath{\left(1-t\right)}}^{n-i-1}=n\text{·}\left[B_{i-1}^{n-1}(t)-B_{i}^{n-1}(t)\right]$
\par\end{center}

Para obtener el máximo, igualamos a cero:

\begin{center}
$B_{i}^{n}(t)'=\left(\begin{array}{c}
n\\
i
\end{array}\right)\text{·}\left[i\text{·}t^{i-1}\text{·\ensuremath{\left(1-t\right)}}^{n-i}-t^{i}\text{·\ensuremath{\left(n-i\right)\text{·}\left(1-t\right)}}^{n-i-1}\right]=\left(\begin{array}{c}
n\\
i
\end{array}\right)\text{·}t^{i-1}\text{·}(1-t)^{n-i-1}\text{·}\left[i\text{·}(1-t)-t\text{·\ensuremath{\left(n-i\right)}}\right]=0$
\par\end{center}

\begin{center}
$i\text{·}(1-t)-t\text{·\ensuremath{\left(n-i\right)}}=0\quad\rightarrow\quad i\text{·}(1-t)=t\text{·\ensuremath{\left(n-i\right)}}\quad\rightarrow\quad i=t\text{·}n\quad\rightarrow\quad t=\frac{i}{n}$
\par\end{center}

Para ver que es el máximo y no un mínimo solo hace falta ver que en
los extremos de la función se tiene que \textbf{$B_{i}^{n}(0)=B_{i}^{n}(1)=0$}
(salvo si $i=0\,\acute{o}\,n$, en cuyo caso \textbf{$B_{0}^{n}(0)=1$}
y \textbf{$B_{n}^{n}(1)=1$}, y son los máximos), mientras que \textbf{$B_{i}^{n}(t)>0$
}para\textbf{ $t\epsilon(0,1)$}.
\end{document}
