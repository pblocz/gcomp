%% LyX 2.1.3 created this file.  For more info, see http://www.lyx.org/.
%% Do not edit unless you really know what you are doing.
\documentclass[spanish]{article}
\usepackage[T1]{fontenc}
\usepackage{color}
\usepackage{amstext}

\makeatletter
%%%%%%%%%%%%%%%%%%%%%%%%%%%%%% User specified LaTeX commands.
\usepackage{babel}


\makeatother

\usepackage{babel}
\addto\shorthandsspanish{\spanishdeactivate{~<>}}

\begin{document}

\title{\textbf{Práctica 5}}


\author{Pablo Cabeza García y Diego González Domínguez}

\maketitle

\section*{Ejercicio:}

\textbf{a) Dada la secuencia de nodos t = Z, calcúlense las funciones
B-spline }$B_{i,k}$ \textbf{correspondientes:}

Si t = Z, entonces $t_{i}=i$, por lo que la expresión de los $B_{i,k}$
es:

\begin{equation}
B_{i,1}(t)=\left\lbrace \begin{array}{lc}
1 & ,\:i\leq t<i+1\\
0 & ,\:en\:otro\:caso
\end{array}\right.
\end{equation}


\begin{equation}
B_{i,k}(t)=\frac{t-i}{k-1}B_{i,k-1}(t)+\frac{i+k-t}{k-1}B_{i+1,k-1}(t)
\end{equation}


es decir, $B_{i,k}(t)$ es distinta de cero en $i\leq t<i+k$

Además, se observa que $B_{i,1}(t)=B_{0,1}(t-i)$, por lo que recursivamente
se tiene que $B_{i,k}(t)=B_{0,k}(t-i)$, de manera que:

\begin{equation}
B_{i,k}(t)=\frac{t-i}{k-1}B_{0,k-1}(t-i)+\frac{i+k-t}{k-1}B_{0,k-1}(t-i-1)
\end{equation}


Por lo que para obtener los $B_{0,k}(t)$ solo se necesita calcular
previamente los $B_{0,j}(t)$ para $j\epsilon\left\{ 1,2,...,k-1\right\} $,
ahorrando mucho trabajo y tiempo. Un ejemplo, para j = 1, 2 y 3:

$B_{0,1}(t)=\left\lbrace \begin{array}{lc}
1 & ,\:0\leq t<1\\
0 & ,\:en\:otro\:caso
\end{array}\right.$

$B_{0,2}(t)=t\text{·}B_{0,1}(t)\mathbf{+}\left(2-t\right)\text{·}B_{1,1}(t)=t\text{·}B_{0,1}(t)\mathbf{+}\left(2-t\right)\text{·}B_{0,1}(t-1)=$

$\qquad=t\text{·}\left\lbrace \begin{array}{lc}
1 & ,\:0\leq t<1\\
0 & ,\:en\:otro\:caso
\end{array}\right.+\left(2-t\right)\text{·}\left\lbrace \begin{array}{lc}
1 & ,\:1\leq t<2\\
0 & ,\:en\:otro\:caso
\end{array}\right.=\left\lbrace \begin{array}{lc}
t & ,\:0\leq t<1\\
2-t & ,\:1\leq t<2\\
0 & ,\:en\:otro\:caso
\end{array}\right.$

$B_{0,3}(t)=\frac{t}{2}\text{·}B_{0,2}(t)\mathbf{+}\frac{3-t}{2}\text{·}B_{1,2}(t)=\frac{t}{2}\text{·}B_{0,2}(t)\mathbf{+}\frac{3-t}{2}\text{·}B_{0,2}(t-1)=$

$\qquad=\frac{t}{2}\text{·}\left\lbrace \begin{array}{lc}
t & ,\:0\leq t<1\\
2-t & ,\:1\leq t<2\\
0 & ,\:en\:otro\:caso
\end{array}\right.+\frac{3-t}{2}\text{·}\left\lbrace \begin{array}{lc}
t-1 & ,\:1\leq t<2\\
3-t & ,\:2\leq t<3\\
0 & ,\:en\:otro\:caso
\end{array}\right.=\left\lbrace \begin{array}{lc}
\frac{t^{2}}{2} & ,\:0\leq t<1\\
\frac{-2t^{2}+6t-3}{2} & ,\:1\leq t<2\\
\frac{(3-t)^{2}}{2} & ,\:2\leq t<3\\
0 & ,\:en\:otro\:caso
\end{array}\right.$

$\vdots$


\section*{\textit{\textcolor{white}{\normalsize{}{}}}}

\textbf{b) Dada una secuencia de nodos arbitraria t, demuéstrese que
si $p$ es un polinomio de grado 1, entonces $p(t)=\sum B_{ik}\text{·}p(t_{i}^{*})$,
siendo $t_{i}^{*}=\left(t_{i+1}+...+t_{i+k-1}\right)/\left(k-1\right)$:}

Por la identidad de Marsden, tenemos que:

\begin{equation}
\frac{(x-t)^{k-1}}{(k-1)!}=\sum_{i=1}\psi_{i}(x)\text{·}B_{i,k}(t)
\end{equation}


con $\psi_{i}(x)=\frac{(x-t_{i+1})\text{·...·}(x-t_{i+k-1})}{(k-1)!}$.

Derivando \textbf{(4)} $s-1$ veces respecto a $x$, se obtiene el
siguiente polinomio:

\begin{equation}
p(t)=\frac{(x-t)^{k-s}}{(k-s)!}=\sum_{i=1}\psi_{i}^{(s-1)}(x)\text{·}B_{i,k}(t)
\end{equation}


Como el enunciado nos dice que p tiene grado 1, eso quiere decir que
$s=k-1$:

\begin{equation}
p(t)=x-t=\sum_{i=1}\left(x-\frac{t_{i+1}+...+t_{i+k-1}}{k-1}\right)\text{·}B_{i,k}(t)=\sum_{i=1}p(t_{i}^{*})\text{·}B_{i,k}(t)
\end{equation}


Esto tambien sería aplicable si $p(t)$ fuera de la forma $p(t)=a\text{·}t+b$,
por lo que queda demostrado.
\end{document}
